\documentclass[a4paper]{article}

\usepackage[francais]{babel}
\usepackage[utf8]{inputenc}
\usepackage[T1]{fontenc} 
\usepackage{fullpage}
\usepackage{graphicx}
\usepackage{xspace}

\usepackage{amsmath}
\usepackage{amsfonts}
\usepackage{amssymb}

\usepackage{hyperref}

%% Tikz for picture drawing %%
\usepackage{tikz}
\usetikzlibrary{arrows}

\usepackage{xcolor,colortbl}
\usepackage{tabularx}
\usepackage{caption}
\usepackage{subcaption}
\usepackage{float}
\usepackage{subfloat}

\graphicspath{{figures/}}

\providecommand{\resume}[1]{\textbf{\textit{Résumé---#1}}}
\providecommand{\keywords}[1]{\textbf{\textit{Mots-clés---#1}}}

\newcommand{\TODO}{\textcolor{red}{\textbf{TODO}}}
\newcommand{\AB}{\emph{Alpha-Bêta}}

\newcommand\q[1]{%(question~#1)}
}
\newcommand\qs[1]{%(questions~#1)}
}
\newcommand\C[1]{\texttt{#1}}

\setlength{\parskip}{1em}


\begin{document}


\title{ARCSYS2 Projet 2%
	\\ Communiquer et jouer en réseau}
\author{Simon Bihel, Florestan De Moor%
	 \\ ENS Rennes, 1ère année Département Informatique et Télécommunications}
\date{05 Avril 2016}

\maketitle

\resume{Ce projet consiste à porter le jeu des 7 couleurs sur un serveur. Ainsi, des clients pourront se connecter au serveur pour assister à la partie et la retransmettre, ou bien pour jouer à distance.}\\

\keywords{Serveur ; Client ; Socket ; Réseau}

\section*{Introduction}


Nous avons dans un projet précédent programmé un jeu des 7 couleurs en langage 
\texttt{C}, permettant à des joueurs humains et des joueurs artificiels de 
s'affronter. Nous avions pour cela implémenté différentes stratégies 
utilisables par des intelligences artificielles, et nous avions comparé leurs 
efficacités à travers un tournoi.

Mais tel que nous l'avons implémenté, si deux joueurs humains veulent jouer, 
ils doivent le faire sur la même machine. Si un tiers veut regarder la partie 
en tant que spectateur, il doit être physiquement présent devant l'écran de la 
machine sur laquelle tourne le jeu. Ceci est peu pratique, et ne permet pas une 
diffusion à large échelle des informations.

C'est pourquoi nous allons à travers ce projet modifier notre code source afin 
de le rendre compatible à une utilisation par un réseau. Le but est de faire 
tourner le jeu sur un serveur, et d'avoir des clients qui peuvent se connecter 
au serveur soit pour jouer, soit pour assister au match et le retransmettre.

Dans un premier temps, on s'intéresse à l'ajout de spectateurs, puis on fait jouer des joueurs à distance, et enfin, on modifie le serveur pour qu'il impose des contraintes équitables aux joueurs.

\section{Ajout de spectateurs}


Pour assister à un match en cours sur un serveur, on utilise la fonction \texttt{spectate()}. Cette fonction suit les différentes étapes suivantes :

%
\begin{itemize}
	\setlength\itemsep{0.5em}
	\item On ouvre un socket client, et on se connecte au serveur.
	\item On envoie une requête \texttt{SPECTATE\_REQUEST} pour signifier au serveur qu'on se connecte en tant que spectateur.
	\item On reçoit du serveur l'état actuel du jeu, la chaîne de caractère suit le format suivant :

		\bgroup
		\def\arraystretch{1.5}
		\begin{center}
		\begin{tabular}{|c|c|}
			\hline 
			\texttt{board\_size[40]} & \texttt{board[board\_size * board\_size]} \\ 
			\hline 
		\end{tabular} 	
		\end{center}
		\egroup
	
	\item On lance alors une partie qui démarre de la configuration reçue, en local. Le serveur envoie un coup sous la forme d'une chaîne de deux caractères. Le premier est la valeur du coup (donc une couleur), et le deuxième le symbole du joueur jouant ce coup. On met à jour le jeu local selon ces informations, et on affiche l'état courant du jeu. 
	\item Lorsque la partie est détectée en local comme étant terminée, on 
	affiche le résultat final, et le socket client se ferme. \\

\end{itemize}
%


\noindent Côté serveur, le protocole est un peu plus compliqué, puisqu'on veut pouvoir 
gérer potentiellement plusieurs spectateurs, dont certains qui arrivent en 
cours de partie. \\

\noindent Regardons d'abord comment le début d'une partie se déroule.
\begin{itemize}
	\setlength\itemsep{0.5em}
	\item On lance le serveur et attend de recevoir des connections pendant 
	quelques secondes.
	\item Lorsqu'on reçoit une connexion, on l'ajoute à un tableau qui garde en 
	mémoire tous les clients connectés.
	\item À chaque fois qu'on reçoit une connexion, on regarde si on a reçu un 
	message pour demander à regarder la partie (\texttt{SPECTATE\_REQUEST}), si c'est le cas alors on ajoute 
	le dernier client accepté dans un second tableau qui garde en mémoire les 
	observateurs.
	\item Une fois qu'on arrête d'attendre, on lance la partie. D'abord on 
	envoie le plateau initial à tous les observateurs.
	\item Pour chaque coup joué on envoie la couleur jouée ainsi que le 
	l'identifiant du joueur qui vient de jouer. \\
\end{itemize}

\noindent Pour avoir des observateurs qui arrivent en milieu de partie, à chaque fois
qu'on envoie le coup qui vient d'être joué aux observateurs, on voit s'il y a
une connexion en attente et si ce nouveau client veut regarder la partie. Si c'est le cas, alors 
on l'ajoute au tableau des observateurs et on lui envoie l'état actuel du
plateau.



\section{Joueurs à distance}

\subsection{Faire jouer des clients}

On souhaite avoir un client qui peut jouer au jeu à distance en se connectant au serveur. On suit le schéma suivant :

%
\begin{itemize}
	\setlength\itemsep{1em}
	\item Le client envoie au serveur un code spécifique signifiant qu'il veut jouer, c'est une requête pour jouer.
	\item Il reçoit la réponse du serveur. Si c'est non (typiquement si la partie est déjà commencée, où s'il y a déjà deux joueurs), alors la demande est refusée, et le client est basculé en mode spectateur. Si la réponse est oui, la demande est acceptée, et le serveur a enregistré le client comme étant joueur.
	\item Le client reçoit une demande de stratégie du serveur. Il choisit sa stratégie, et l'envoie au serveur.
	\item Le serveur envoie ensuite le tableau initial du jeu, pour commencer à jouer.
	\item Le serveur envoie au client l'information suivante : est-ce que le client joue en premier ou non ?
	\item Le jeu tourne sur le serveur, et celui-ci envoie des requêtes de coup au client lorsque c'est nécessaire. Le client fait tourner en local un jeu afin de pouvoir calculer son coup suivant, et il l'envoie au serveur.
	\item Le serveur envoie au client les coups de l'adversaire, pour qu'il puisse mettre à jour sa version locale du jeu.
\end{itemize}
%


\subsection{Robustesse}

Il est possible que lors d'une partie, un joueur déconnecte soudainement ou 
bien que le serveur tombe en panne. Il faut ainsi ajouter des protocoles de 
protection pour savoir comment réagir dans ces cas là.

Si un spectateur déconnecte pendant la partie, le serveur ne doit pas être 
perturbé. Il peut cependant signaler par un affichage qu'il a remarqué une 
déconnexion d'un spectateur.

Si un joueur déconnecte \TODO not supported yet

Du point de vue de la partie client, toutes les interactions avec le serveur 
(avec les fonctions \texttt{send} et \texttt{recv}) sont cachées derrière deux 
fonctions \texttt{server\_to\_client} et \texttt{client\_to\_server}. Ces 
fonctions vérifient la valeur de retour, et utilisent une boucle 
\texttt{while}. Un compteur est initialisé à \texttt{MAX\_SERVER\_MISS} et un 
temps d'attente à une seconde. Si on ne reçoit rien de valide du serveur 
(valeur de retour négative ou nulle), on décrémente le compteur, on attend, et 
on double la valeur du temps d'attente, avant d'entrer dans la prochaine 
itération de la boucle. Obtenir une réponse positive du serveur fait sortir de 
la boucle. Une fois le compteur nul, on sort de la boucle, et on quitte le 
programme en expliquant que le serveur semble être déconnecté. On effectue 
ainsi \texttt{MAX\_SERVER\_MISS} demandes au serveur, espacées d'un temps qui 
augmente exponentiellement. Concrètement, cela signifie que si l'on a pas 
réussi à avoir de connexion avec le serveur au bout de 
$2^{\texttt{MAX\_SERVER\_MISS}} - 1$ secondes, alors on le considère déconnecté.


\subsection{Limite de temps}

Calcul du prochain coups dans un precessus fils, au bout d'un certain temps on 
le tue et on voit si il a réussi a calculé le coup, sinon c'est deja trop tard 
et on attend le coup joué par le serveur à notre place.


\section{Contraintes imposées par le serveur}



\section{Critique}

Pour une partie il faut avoir 2 threads (pour le serveur comme pour le joueur) 
avec un qui gère les interactions serveur/client et un autre qui gère le jeu 
(une sort d'automate). Le thread jeu dors et est réveillé des qu'un des 
elements du jeu est changé.


\section*{Conclusion}

\TODO


\bibliographystyle{plain} 
\bibliography{report_arcsys2p2}

\end{document}